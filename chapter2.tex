\chapter{ВОЖАТЫЙ}

\hspace{0.5\textwidth}
\parbox{ Сторона ль моя, сторонушка,\\
Сторона незнакомая!\\
Что не сам ли я на тебя зашел,\\
Что не добрый ли да меня конь завез:\\
Завезла меня, доброго молодца,\\
Прытость, бодрость молодецкая\\
И хмелинушка кабацкая.\\
Старинная песня.}


Дорожные размышления мои были не очень приятны. Проигрыш мой, по тогдашним ценам, был немаловажен. Я не мог не признаться в душе, что поведение мое в симбирском трактире было глупо, и чувствовал себя виноватым перед Савельичем. Все это меня мучило. Старик угрюмо сидел на облучке, отворотясь от меня, и молчал, изредка только покрякивая. Я непременно хотел с ним помириться и не знал с чего начать. Наконец я сказал ему: «Ну, ну, Савельич! полно, помиримся, виноват; вижу сам, что виноват. Я вчера напроказил, а тебя напрасно обидел. Обещаюсь вперед вести себя умнее и слушаться тебя. Ну, не сердись; помиримся».

— Эх, батюшка Петр Андреич! — отвечал он с глубоким вздохом. — Сержусь-то я на самого себя; сам я кругом виноват. Как мне было оставлять тебя одного в трактире! Что делать? Грех попутал: вздумал забрести к дьячихе, повидаться с кумою. Так-то: зашел к куме, да засел в тюрьме. Беда да и только!.. Как покажусь я на глаза господам? что скажут они, как узнают, что дитя пьет и играет.

Чтоб утешить бедного Савельича, я дал ему слово впредь без его согласия не располагать ни одною копейкою. Он мало-помалу успокоился, хотя все еще изредка ворчал про себя, качая головою: «Сто рублей! легко ли дело!»

Я приближался к месту моего назначения. Вокруг меня простирались печальные пустыни, пересеченные холмами и оврагами. Все покрыто было снегом. Солнце садилось. Кибитка ехала по узкой дороге, или точнее по следу, проложенному крестьянскими санями. Вдруг ямщик стал посматривать в сторону и наконец, сняв шапку, оборотился ко мне и сказал:

— Барин, не прикажешь ли воротиться?

— Это зачем?

— Время ненадежно: ветер слегка подымается; вишь, как он сметает порошу.

— Что ж за беда!

— А видишь там что? (Ямщик указал кнутом на восток.)

— Я ничего не вижу, кроме белой степи да ясного неба.

— А вон — вон: это облачко.

Я увидел в самом деле на краю неба белое облачко, которое принял было сперва за отдаленный холмик. Ямщик изъяснил мне, что облачко предвещало буран.
Я слыхал о тамошних метелях и знал, что целые обозы бывали ими занесены. Савельич, согласно со мнением ямщика, советовал воротиться. Но ветер показался мне не силен; я понадеялся добраться заблаговременно до следующей станции и велел ехать скорее.

Ямщик поскакал; но все поглядывал на восток. Лошади бежали дружно. Ветер между тем час от часу становился сильнее. Облачко обратилось в белую тучу, которая тяжело подымалась, росла и постепенно облегала небо. Пошел мелкий снег — и вдруг повалил хлопьями. Ветер завыл; сделалась метель. В одно мгновение темное небо смешалось со снежным морем. Все исчезло. «Ну, барин, — закричал ямщик, — беда: буран!»...

Я выглянул из кибитки: все было мрак и вихорь. Ветер выл с такой свирепой выразительностию, что казался одушевленным; снег засыпал меня и Савельича; лошади шли шагом — и скоро стали. «Что же ты не едешь?» — спросил я ямщика с нетерпением. «Да что ехать? — отвечал он, слезая с облучка, — невесть и так куда заехали: дороги нет, и мгла кругом». Я стал было его бранить. Савельич за него заступился. «И охота было не слушаться, — говорил он сердито, — воротился бы на постоялый двор, накушался бы чаю, почивал бы себе до утра, буря б утихла, отправились бы далее. И куда спешим? Добро бы на свадьбу!» Савельич был прав. Делать было нечего. Снег так и валил. Около кибитки подымался сугроб. Лошади стояли, понуря голову и изредка вздрагивая. Ямщик ходил кругом, от нечего делать улаживая упряжь. Савельич ворчал; я глядел во все стороны, надеясь увидеть хоть признак жила или дороги, но ничего не мог различить, кроме мутного кружения метели... Вдруг увидел я что-то черное. «Эй, ямщик! — закричал я, — смотри: что там такое чернеется?» Ямщик стал всматриваться. «А бог знает, барин, — сказал он, садясь на свое место, — воз не воз, дерево не дерево, а кажется, что шевелится. Должно быть, или волк, или человек».

Я приказал ехать на незнакомый предмет, который тотчас и стал подвигаться нам навстречу. Через две минуты мы поравнялись с человеком.
— Гей, добрый человек! — закричал ему ямщик. — Скажи, не знаешь ли где дорога?
— Дорога-то здесь; я стою на твердой полосе, — отвечал дорожный, — да что толку?
— Послушай, мужичок, — сказал я ему, — знаешь ли ты эту сторону? Возьмешься ли ты довести меня до ночлега?

— Сторона мне знакомая, — отвечал дорожный, — слава богу, исхожена и изъезжена вдоль и поперек. Да, вишь, какая погода: как раз собьешься с дороги. Лучше здесь остановиться да переждать, авось буран утихнет да небо прояснится: тогда найдем дорогу по звездам.

Его хладнокровие ободрило меня. Я уж решился, предав себя божией воле, ночевать посреди степи, как вдруг дорожный сел проворно на облучок и сказал ямщику: «Ну, слава богу, жило недалеко; сворачивай вправо да поезжай».

— А почему мне ехать вправо? — спросил ямщик с неудовольствием. — Где ты видишь дорогу? Небось: лошади чужие, хомут не свой, погоняй не стой. — Ямщик казался мне прав. «В самом деле, — сказал я, — почему думаешь ты, что жило недалече?» — «А потому, что ветер оттоле потянул, — отвечал дорожный, — и я слышу, дымом пахнуло; знать, деревня близко». Сметливость его и тонкость чутья меня изумили. Я велел ямщику ехать. Лошади тяжело ступали по глубокому снегу. Кибитка тихо подвигалась, то въезжая на сугроб, то обрушаясь в овраг и переваливаясь то на одну, то на другую сторону. Это похоже было на плавание судна по бурному морю. Савельич охал, поминутно толкаясь о мои бока. Я опустил циновку, закутался в шубу и задремал, убаюканный пением бури и качкою тихой езды.

Мне приснился сон, которого никогда не мог я позабыть и в котором до сих пор вижу нечто пророческое, когда соображаю с ним странные обстоятельства моей жизни. Читатель извинит меня: ибо, вероятно, знает по опыту, как сродно человеку предаваться суеверию, несмотря на всевозможное презрение к предрассудкам.
Я находился в том состоянии чувств и души, когда существенность, уступая мечтаниям, сливается с ними в неясных видениях первосония. Мне казалось, буран еще свирепствовал и мы еще блуждали по снежной пустыне... Вдруг увидел я вороты и въехал на барский двор нашей усадьбы. Первою мыслию моею было опасение, чтобы батюшка не прогневался на меня за невольное возвращение под кровлю родительскую и не почел бы его умышленным ослушанием. С беспокойством я выпрыгнул из кибитки и вижу: матушка встречает меня на крыльце с видом глубокого огорчения. «Тише, — говорит она мне, — отец болен при смерти и желает с тобою проститься». Пораженный страхом, я иду за нею в спальню. Вижу, комната слабо освещена; у постели стоят люди с печальными лицами. Я тихонько подхожу к постеле; матушка приподымает полог и говорит: «Андрей Петрович, Петруша приехал; он воротился, узнав о твоей болезни; благослови его». Я стал на колени и устремил глаза мои на больного. Что ж?.. Вместо отца моего вижу в постеле лежит мужик с черной бородою, весело на меня поглядывая. Я в недоумении оборотился к матушке, говоря ей: «Что это значит? Это не батюшка. И к какой мне стати просить благословения у мужика?» — «Все равно, Петруша, — отвечала мне матушка, — это твой посажёный отец; поцелуй у него ручку, и пусть он тебя благословит...» Я не соглашался. Тогда мужик вскочил с постели, выхватил топор из-за спины и стал махать во все стороны. Я хотел бежать... и не мог; комната наполнилась мертвыми телами; я спотыкался о тела и скользил в кровавых лужах... Страшный мужик ласково меня кликал, говоря: «Не бойсь, подойди под мое благословение...» Ужас и недоумение овладели мною... И в эту минуту я проснулся; лошади стояли; Савельич дергал меня за руку, говоря: «Выходи, сударь: приехали».

— На постоялый двор. Господь помог, наткнулись прямо на забор. Выходи, сударь, скорее да обогрейся.

Я вышел из кибитки. Буран еще продолжался, хотя с меньшею силою. Было так темно, что хоть глаз выколи. Хозяин встретил нас у ворот, держа фонарь под полою, и ввел меня в горницу, тесную, но довольно чистую; лучина освещала ее. На стене висела винтовка и высокая казацкая шапка.

Хозяин, родом яицкий казак, казался мужик лет шестидесяти, еще свежий и бодрый. Савельич внес за мною погребец, потребовал огня, чтоб готовить чай, который никогда так не казался мне нужен. Хозяин пошел хлопотать.
— Где же вожатый? — спросил я у Савельича.

«Здесь, ваше благородие», — отвечал мне голос сверху. Я взглянул на полати и увидел черную бороду и два сверкающие глаза. «Что, брат, прозяб?» — «Как не прозябнуть в одном худеньком армяке! Был тулуп, да что греха таить? заложил вечор у целовальника: мороз показался не велик». В эту минуту хозяин вошел с кипящим самоваром; я предложил вожатому нашему чашку чаю; мужик слез с полатей. Наружность его показалась мне замечательна: он был лет сорока, росту среднего, худощав и широкоплеч. В черной бороде его показывалась проседь; живые большие глаза так и бегали. Лицо его имело выражение довольно приятное, но плутовское. Волоса были обстрижены в кружок; на нем был оборванный армяк и татарские шаровары. Я поднес ему чашку чаю; он отведал и поморщился. «Ваше благородие, сделайте мне такую милость, — прикажите поднести стакан вина; чай не наше казацкое питье». Я с охотой исполнил его желание. Хозяин вынул из ставца штоф и стакан, подошел к нему и, взглянув ему в лицо: «Эхе, — сказал он, — опять ты в нашем краю! Отколе бог принес?» Вожатый мой мигнул значительно и отвечал поговоркою: «В огород летал, конопли клевал; швырнула бабушка камушком — да мимо. Ну, а что ваши?»

«Да что наши! — отвечал хозяин, продолжая иносказательный разговор. — Стали было к вечерне звонить, да попадья не велит: поп в гостях, черти на погосте». — «Молчи, дядя, — возразил мой бродяга, — будет дождик, будут и грибки; а будут грибки, будет и кузов. А теперь (тут он мигнул опять) заткни топор за спину: лесничий ходит. Ваше благородие! за ваше здоровье!» При сих словах он взял стакан, перекрестился и выпил одним духом. Потом поклонился мне и воротился на полати.

Я ничего не мог тогда понять из этого воровского разговора; но после уж догадался, что дело шло о делах Яицкого войска, в то время только что усмиренного после бунта 1772 года. Савельич слушал с видом большого неудовольствия. Он посматривал с подозрением то на хозяина, то на вожатого. Постоялый двор, или, по-тамошнему, умет, находился в стороне, в степи, далече от всякого селения, и очень походил на разбойническую пристань. Но делать было нечего. Нельзя было и подумать о продолжении пути. Беспокойство Савельича очень меня забавляло. Между тем я расположился ночевать и лег на лавку. Савельич решился убраться на печь; хозяин лег на полу. Скоро вся изба захрапела, и я заснул как убитый.
Проснувшись поутру довольно поздно, я увидел, что буря утихла. Солнце сияло. Снег лежал ослепительной пеленою на необозримой степи. Лошади были запряжены. Я расплатился с хозяином, который взял с нас такую умеренную плату, что даже Савельич с ним не заспорил и не стал торговаться по своему обыкновению, и вчерашние подозрения изгладились совершенно из головы его. Я позвал вожатого, благодарил за оказанную помочь и велел Савельичу дать ему полтину на водку. Савельич нахмурился. «Полтину на водку! — сказал он, — за что это? За то, что ты же изволил подвезти его к постоялому двору? Воля твоя, сударь: нет у нас лишних полтин. Всякому давать на водку, так самому скоро придется голодать». Я не мог спорить с Савельичем. Деньги, по моему обещанию, находились в полном его распоряжении. Мне было досадно, однако ж, что не мог отблагодарить человека, выручившего меня если не из беды, то по крайней мере из очень неприятного положения. «Хорошо, — сказал я хладнокровно, — если не хочешь дать полтину, то вынь ему что-нибудь из моего платья. Он одет слишком легко. Дай ему мой заячий тулуп».

— Помилуй, батюшка Петр Андреич! — сказал Савельич. — Зачем ему твой заячий тулуп? Он его пропьет, собака, в первом кабаке.

— Это, старинушка, уж не твоя печаль, — сказал мой бродяга, — пропью ли я или нет. Его благородие мне жалует шубу со своего плеча: его на то барская воля, а твое холопье дело не спорить и слушаться.

— Бога ты не боишься, разбойник! — отвечал ему Савельич сердитым голосом. — Ты видишь, что дитя еще не смыслит, а ты и рад его обобрать, простоты его ради. Зачем тебе барский тулупчик? Ты и не напялишь его на свои окаянные плечища.

— Прошу не умничать, — сказал я своему дядьке, — сейчас неси сюда тулуп.

— Господи владыко! — простонал мой Савельич. — Заячий тулуп почти новешенький! и добро бы кому, а то пьянице оголелому!

Однако заячий тулуп явился. Мужичок тут же стал его примеривать. В самом деле тулуп, из которого успел и я вырасти, был немножко для него узок. Однако он кое-как умудрился и надел его, распоров по швам. Савельич чуть не завыл, услышав, как нитки затрещали. Бродяга был чрезвычайно доволен моим подарком. Он проводил меня до кибитки и сказал с низким поклоном: «Спасибо, ваше благородие! Награди вас господь за вашу добродетель. Век не забуду ваших милостей». Он пошел в свою сторону, а я отправился далее, не обращая внимания на досаду Савельича, и скоро позабыл о вчерашней вьюге, о своем вожатом и о заячьем тулупе.
Приехав в Оренбург, я прямо явился к генералу. Я увидел мужчину росту высокого, но уже сгорбленного старостию. Длинные волосы его были совсем белы. Старый полинялый мундир напоминал воина времен Анны Иоанновны, а в его речи сильно отзывался немецкий выговор. Я подал ему письмо от батюшки. При имени его он взглянул на меня быстро: «Поже мой! — сказал он. — Тавно ли, кажется, Андрей Петрович был еще твоих лет, а теперь вот уш какой у него молотец! Ах, фремя, фремя!» Он распечатал письмо и стал читать его вполголоса, делая свои замечания. «Милостивый государь Андрей Карлович, надеюсь, что ваше превосходительство»... Это что за серемонии? Фуй, как ему не софестно! Конечно: дисциплина перво дело, но так ли пишут к старому камрад?.. «ваше превосходительство не забыло»... гм... «и... когда... покойным фельдмаршалом Мин... походе... также и... Каролинку»... Эхе, брудер! так он еще помнит стары наши проказ? «Теперь о деле... К вам моего повесу»... гм... «держать в ежовых рукавицах»... Что такое ешовы рукавиц? Это, должно быть, русска поговорк... Что такое «дершать в ешовых рукавицах?» — повторил он, обращаясь ко мне.

— Это значит, — отвечал я ему с видом как можно более невинным, — обходиться ласково, не слишком строго, давать побольше воли, держать в ежовых рукавицах.

— Гм, понимаю... «и не давать ему воли»... нет, видно ешовы рукавицы значит не то... «При сем... его паспорт»... Где же он? А, вот... «отписать в Семеновский»... Хорошо, хорошо: все будет сделано... «Позволишь без чинов обнять себя и... старым товарищем и другом» — а! наконец догадался... и прочая и прочая... Ну, батюшка, — сказал он, прочитав письмо и отложив в сторону мой паспорт, — все будет сделано: ты будешь офицером переведен в *** полк, и, чтоб тебе времени не терять, то завтра же поезжай в Белогорскую крепость, где ты будешь в команде капитана Миронова, доброго и честного человека. Там ты будешь на службе настоящей, научишься дисциплине. В Оренбурге делать тебе нечего; рассеяние вредно молодому человеку. А сегодня милости просим: отобедать у меня».
«Час от часу не легче! — подумал я про себя, — к чему послужило мне то, что еще в утробе матери я был уже гвардии сержантом! Куда это меня завело? В *** полк и в глухую крепость на границу киргиз-кайсацких степей!..» Я отобедал у Андрея Карловича, втроем с его старым адъютантом. Строгая немецкая экономия царствовала за его столом, и я думаю, что страх видеть иногда лишнего гостя за своею холостою трапезою был отчасти причиною поспешного удаления моего в гарнизон. На другой день я простился с генералом и отправился к месту моего назначения.